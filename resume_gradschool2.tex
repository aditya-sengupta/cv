\documentclass[paper=a4,fontsize=10pt]{scrartcl}
							
\usepackage[english]{babel}
\usepackage[utf8x]{inputenc}
\usepackage[protrusion=true,expansion=true]{microtype}
\usepackage{comment}
\usepackage{lipsum}
\usepackage{amsmath,amsfonts,amsthm}     % Math packages
\usepackage{graphicx}                    % Enable pdflatex
\usepackage[svgnames]{xcolor}            % Colors by their 'svgnames'
\usepackage[left=0.5in,right=0.5in,top=0.5in,bottom=0.1in]{geometry}
	%\textheight=600px                    % Saving trees ;-)
\usepackage{url}
\usepackage{comment}

\pagestyle{empty}           

\usepackage{sectsty}


\sectionfont{%			            % Change font of \section command
	%\usefont{OT1}{phv}{b}{n}%		% bch-b-n: CharterBT-Bold font
	\sectionrule{0pt}{0pt}{-5pt}{1pt}
}


%\newlength{\spacebox}
%\settowidth{\spacebox}{8888888888}			% Box to align text
%\newcommand{\sepspace}{\vspace*{1em}}		% Vertical space macro

\newcommand{\MyName}[1]{ % Name
                \Huge \hfill #1
                \par \normalsize \normalfont}

\newcommand{\NewPart}[1]{\section*{{#1}}}
		
\newcommand{\EducationEntry}[4]{
		\noindent \textbf{#1} \hfill      % Study
		\colorbox{White}{%
			\parbox{5cm}{%
			\hfill\color{Black}#2}} \par  % Duration
		\noindent #3 \par        % School
		\noindent\hangindent=2em\hangafter=0 \small #4 % Description
		\normalsize \par}
		
\begin{document}

\MyName{Aditya Rohan Sengupta}

\vspace{0.4cm}
\noindent\textbf{Email:} \url{aditya.sengupta@berkeley.edu} \hspace{0.5cm} \textbf{Website:} \url{aditya-sengupta.github.io} \hspace{0.5cm} \textbf{Phone :} +1 (510) 697 6549
\vspace{-0.5cm}
\NewPart{Education}{}

\noindent\EducationEntry{University of Cambridge; Master of Advanced Study in Applied Mathematics}{Oct 2021-Jun 2022}{Part III of the Mathematical Tripos; Girton College, Cambridge} 

\noindent\EducationEntry{University of California, Berkeley; Bachelor of Science}{Aug 2017-May 2021}{Engineering Physics/Engineering Mathematics and Statistics double major; GPA 3.822}{\noindent\textbf{Relevant Coursework (graduate-level*)}\\\underline{Math/Stat} - Probability \& Random Processes, Differentiable Manifolds*, Abstract Algebra, Analysis, Linear Algebra\\\underline{EECS} - Numerical Simulation*, Information Theory*, Machine Learning, Control Systems, Signals, Optical Engineering\\ \underline{Physics} - Advanced Experimentation, Quantum Mechanics, Analytic Mechanics, E\&M/Optics, Cosmology, Statistical Mechanics}
\noindent

\vspace{-0.5cm}
\NewPart{Experience}{}

\EducationEntry{Lamat Fellow, UC Santa Cruz Astronomy}{May-August 2021}{
\underline{System Identification for Tip-Tilt Mirrors}, with Prof. Rebecca Jensen-Clem and Dr. Benjamin Gerard
}

\EducationEntry{Researcher, UC Berkeley Astronomy and EECS}{May 2018-present}{
\vspace{-0.3cm}
\noindent\underline{Exoplanet Probabilistic Modeling with \textit{TESS}}, with Prof. Courtney Dressing (ongoing)
\begin{itemize}
    \setlength\itemsep{0.1em}
    \item Applying Approximate Bayesian Computation to data from NASA's Transiting Exoplanet Survey Satellite for exoplanet occurrence estimates.
    \vspace{-0.15cm}
    \item Creating framework to better correct for telescope optics in \textit{TESS} imaging data, for improved exoplanet detection.
    \vspace{-0.15cm}
    \item Analyzing TESS instrument systematics to build a model for the probability of observing a transit.
    \vspace{-0.15cm}
\end{itemize}

\noindent\underline{Optimal Tip-Tilt Correction for Adaptive Optics}, with Dr. Rebecca Jensen-Clem (now Asst. Prof, UC Santa Cruz)
\begin{itemize}
\vspace{-0.3cm}
\setlength\itemsep{-0.3em}
    \item Simulated control schemes in tip-tilt wavefront correction, verified with telemetry data from Keck telescopes.
    \item Showed improvements over empirical integrator through physics-based model predictive control incorporating hardware loop delays, atmospheric turbulence, vibrations.
    %\item Simulation verification complete; real-life testing underway at new UC Berkeley adaptive optics test bench.
    %\item Wrote tutorial and reference documentation on tip-tilt correction.
    \item Invited to Center for Adaptive Optics Fall Retreat 2019 to discuss and share work.
\end{itemize}
\underline{Pyramid Wavefront Sensor Simulation for the Keck Telescopes}, with Dr. Rebecca Jensen-Clem
\begin{itemize}
\vspace{-0.15cm}
\setlength\itemsep{-0.3em}
    \item Created simulation of an adaptive optics loop with a pyramid wavefront sensor, newly installed at Keck II.
    \item Demonstrated imaging quality improvements due to predictive control algorithms, to be implemented at Keck.
    \item Conducted testing and QA for HCIPy (High Contrast Imaging for Python) library.
\end{itemize}
}

\EducationEntry{Teaching: UC Berkeley EECS, UC Berkeley Student Learning Center}{Jan 2018-May 2021}{
\begin{itemize}
\vspace{-0.15cm}
\setlength\itemsep{-0.3em}
\item Teaching Assistant for EECS 126, Probability \& Random Processes. Designed lab assignment on Kalman filtering, to visualize and practically implement difficult concepts. Co-wrote official course notes, supplemental material.
\item Multivariable Calculus Study Group Leader: designed worksheets for, taught twice-weekly discussion-style sections.
\item Mathematics tutor for lower-division courses at Berkeley Student Learning Center, for calculus and linear algebra.
\item Tutor/Reader for Probability and Random Processes, Data Structures, Control Systems.
\end{itemize}}

\vspace{-0.15cm}
\EducationEntry{Undergraduate Lab at Berkeley: Curriculum Chair and Mentor}{Aug 2019-May 2021}{
\begin{itemize}
\vspace{-0.15cm}
\setlength\itemsep{-0.3em}
\item Writing and managing lab-wide instructional modules to introduce aspiring researchers to essential skills: programming/Git, research literacy and communication, statistics, PCB design.
\item Leading an independent research team of freshman/sophomore level physics students to construct a Cosmic Microwave Background detector, by taking radiometer observations from a weather balloon.
\item Managing subgroups for detector circuit-board design, mechanical construction, and data denoising and inference algorithms; working individually with mentees to match subprojects to future plans. 
\end{itemize}}

\EducationEntry{Space Technology and Rocketry (STAR) at Berkeley}{Aug 2017-Aug 2019}{
\begin{itemize}
\vspace{-0.15cm}
\setlength\itemsep{-0.3em}
\item Co-lead for new Simulations subteam on UC Berkeley's high-powered rocketry team. Responsible for training members, documenting procedures, liaising with team to produce actionable simulation results.
\item Built first-principles rocket flight simulation and Kalman filter for optimal parachute deployment timing.
%\item Designed/analyzed flight black box, contributed to liquid bipropellant engine model (thermal/fluid).
\item Optimized geometry of airframe transition section for aerodynamics and load bearing via ANSYS simulations.
\end{itemize}}

\noindent
\textbf{Skills}: High-dimensional statistics and MCMC, time-series filtering and modelling, first-principles physical simulations\\
\textbf{Programming languages}: Python, Julia, Java, Matlab, LaTeX\\
\textbf{Engineering tools}: ANSYS Structural, ANSYS CFD, SimScale, Solidworks, Zemax, KiCAD


\end{document}
